\documentclass[]{article}
\usepackage{lmodern}
\usepackage{amssymb,amsmath}
\usepackage{ifxetex,ifluatex}
\usepackage{fixltx2e} % provides \textsubscript
\ifnum 0\ifxetex 1\fi\ifluatex 1\fi=0 % if pdftex
  \usepackage[T1]{fontenc}
  \usepackage[utf8]{inputenc}
\else % if luatex or xelatex
  \ifxetex
    \usepackage{mathspec}
  \else
    \usepackage{fontspec}
  \fi
  \defaultfontfeatures{Ligatures=TeX,Scale=MatchLowercase}
\fi
% use upquote if available, for straight quotes in verbatim environments
\IfFileExists{upquote.sty}{\usepackage{upquote}}{}
% use microtype if available
\IfFileExists{microtype.sty}{%
\usepackage{microtype}
\UseMicrotypeSet[protrusion]{basicmath} % disable protrusion for tt fonts
}{}
\usepackage[margin=1in]{geometry}
\usepackage{hyperref}
\hypersetup{unicode=true,
            pdftitle={Assignment1\_Seshadri},
            pdfauthor={Sri Seshadri},
            pdfborder={0 0 0},
            breaklinks=true}
\urlstyle{same}  % don't use monospace font for urls
\usepackage{graphicx,grffile}
\makeatletter
\def\maxwidth{\ifdim\Gin@nat@width>\linewidth\linewidth\else\Gin@nat@width\fi}
\def\maxheight{\ifdim\Gin@nat@height>\textheight\textheight\else\Gin@nat@height\fi}
\makeatother
% Scale images if necessary, so that they will not overflow the page
% margins by default, and it is still possible to overwrite the defaults
% using explicit options in \includegraphics[width, height, ...]{}
\setkeys{Gin}{width=\maxwidth,height=\maxheight,keepaspectratio}
\IfFileExists{parskip.sty}{%
\usepackage{parskip}
}{% else
\setlength{\parindent}{0pt}
\setlength{\parskip}{6pt plus 2pt minus 1pt}
}
\setlength{\emergencystretch}{3em}  % prevent overfull lines
\providecommand{\tightlist}{%
  \setlength{\itemsep}{0pt}\setlength{\parskip}{0pt}}
\setcounter{secnumdepth}{0}
% Redefines (sub)paragraphs to behave more like sections
\ifx\paragraph\undefined\else
\let\oldparagraph\paragraph
\renewcommand{\paragraph}[1]{\oldparagraph{#1}\mbox{}}
\fi
\ifx\subparagraph\undefined\else
\let\oldsubparagraph\subparagraph
\renewcommand{\subparagraph}[1]{\oldsubparagraph{#1}\mbox{}}
\fi

%%% Use protect on footnotes to avoid problems with footnotes in titles
\let\rmarkdownfootnote\footnote%
\def\footnote{\protect\rmarkdownfootnote}

%%% Change title format to be more compact
\usepackage{titling}

% Create subtitle command for use in maketitle
\newcommand{\subtitle}[1]{
  \posttitle{
    \begin{center}\large#1\end{center}
    }
}

\setlength{\droptitle}{-2em}
  \title{Assignment1\_Seshadri}
  \pretitle{\vspace{\droptitle}\centering\huge}
  \posttitle{\par}
  \author{Sri Seshadri}
  \preauthor{\centering\large\emph}
  \postauthor{\par}
  \predate{\centering\large\emph}
  \postdate{\par}
  \date{4/7/2018}


\begin{document}
\maketitle

\section{Section 2.8 page 59 - Effect of transformation on time series
data.}\label{section-2.8-page-59---effect-of-transformation-on-time-series-data.}

\subsection{a) Monthly total of peole on unemployment benefits in
Australia (Jan 1956 - July
1992)}\label{a-monthly-total-of-peole-on-unemployment-benefits-in-australia-jan-1956---july-1992}

Figure 1 shows the time series of number of people in unemplyment
benefits in Australia by month. The time series is affected by

\begin{verbatim}
    *  Population growth over time
    *  External factors like:
        * state of the economy
        * the benefit provided by the government.
\end{verbatim}

It would be useful to normalize the data by population, to get the
percent unemployed of the total population. Then if need be a
transformation on the normalized data can be made.

\linebreak

\begin{figure}
\centering
\includegraphics{Assignment1_files/figure-latex/unnamed-chunk-1-1.pdf}
\caption{Monthly number of people on unemployment benefits in Australia}
\end{figure}

\subsection{b) Monthly total accidental deaths in the United
States}\label{b-monthly-total-accidental-deaths-in-the-united-states}

The top plot in figure 2 shows the total accidental deaths by month in
the US. There is seasonality in the data, where the total accidents
peaking at July. The variation in the seasonality may be mitigated by
normalizing the totals by dividing by the number of days in the month.

The middle plot in figure 2 shows the normalized total by days in the
months; i.e.~Monthly average accidental deaths per day. We see that
there is some smoothing of the raw data. While there is not much
variation in seasonality, there is interest in further making the size
of the seasonal variation equal across seasons.

The bottom chart of figure 2 shows a box-cox transformation of the
average accidental deaths. Its seen that there isn't much affect as
expected.

\begin{figure}
\centering
\includegraphics{Assignment1_files/figure-latex/unnamed-chunk-2-1.pdf}
\caption{Top: Total accidental deaths by month in the US, Middle:
Monthly Average Accidental deaths per day, Bottom: Transformed monthly
average accidental deaths per day}
\end{figure}

\subsection{c) Quarterly production of bricks (in millions) at Portland,
Australia}\label{c-quarterly-production-of-bricks-in-millions-at-portland-australia}

The top chart in Figure 3 shows the time series plot of quarterly brick
production at Portland Australia. The time series exhibits an increasing
trend and seasonality. The variation increases with time / levels.
Box-Cox transformation is appropriate for this case. The ideal lambda
for the data was 0.255. The bottom chart of figure 3 shows the
transformed data. The variation is better across time, of course the
huge downward spikes is not fully mitigated.

\begin{figure}
\centering
\includegraphics{Assignment1_files/figure-latex/unnamed-chunk-3-1.pdf}
\caption{Top: Quarterly production of bricks (in millions) at Portland,
Australia, Bottom : BoxCox transformed Quarterly production of bricks}
\end{figure}


\end{document}
